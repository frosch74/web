%%%%%%%%%%%%%%%%%%%%%%%%%%%%%%%%%%%%%%%%%%%%%%%%%%%%%%%%%%%%%%%%%%%%%%%%%%%%%%%
%                            CONCEPTION DE SITES WEB 
%  Cours proposé à l'ENSICAEN en spécialité informatique 1ère année (2014-2015)
%  puis au lycée Dorian de Paris en BTS SN-IR 1ère année (2015-2016)
%                                    2014-2015
%                                  Alain Lebret
%%%%%%%%%%%%%%%%%%%%%%%%%%%%%%%%%%%%%%%%%%%%%%%%%%%%%%%%%%%%%%%%%%%%%%%%%%%%%%%
\documentclass[a4paper,12pt]{article}
\usepackage[frenchb]{babel}
\usepackage[T1]{fontenc}
%%%
\newif\ifTRUETYPE % Test pour l'utilisation des polices True Type
\TRUETYPEtrue
%\TRUETYPEfalse
\ifTRUETYPE
\else
  \usepackage[utf8]{inputenc}
\fi
%%%
\usepackage{graphicx}
\usepackage{colortbl}
\usepackage[table]{xcolor}
\usepackage[hmargin=3cm,bottom=5cm]{geometry}
\usepackage{fancyhdr}
\usepackage{amsmath,amsfonts,amssymb}
%%%%%%%%%%%%%%%%%%%%%%%%%%%%%%%%%%%%%%%%%%%%%%%%%%%%%%%%%%%%%%%%%%%%%%%%%%%%%%%
% Utilisation de polices True Type à la place des polices LaTeX traditionnelles
% Le package "inputenc" doit être enlevé et le fichier doit être compilé avec
% XeLaTeX
%%%%%%%%%%%%%%%%%%%%%%%%%%%%%%%%%%%%%%%%%%%%%%%%%%%%%%%%%%%%%%%%%%%%%%%%%%%%%%%
\ifTRUETYPE
\usepackage{fontspec}
\setmainfont{Calibri} 
\setmonofont[SmallCapsFont={Latin Modern Mono Caps}]{Latin Modern Mono Light}
\fi
%%%

\pagestyle{fancy}
\renewcommand{\headrule}{\relax}
\setlength{\headheight}{70pt}
\fancyhf{}
\lhead{}
\rhead{\includegraphics[height=\headheight]{http://frosch75.free.fr/snir1/pictures/logoEcole.jpg}}
\lfoot{\textit{Spécialité informatique - Conception de sites Web}\\\includegraphics[height=0.12\headheight]{http://frosch75.free.fr/snir1/pictures/vague2013seule.jpg}
A. Lebret (2015)}
\rfoot{\thepage}

\definecolor{couleur_code}{RGB}{90,90,90} 
\newcommand\Code[1]{\textcolor{couleur_code}{\texttt{#1}}}
\newcounter{Question}
\setcounter{Question}{1}
\newcommand\Question{\vspace{5pt}\textbf{Question \arabic{Question}\addtocounter{Question}{1}.~~}}
\makeatletter
\newcommand{\Qlabel}[1]{\addtocounter{Question}{-1}\protected@write \@auxout {}{\string \newlabel {#1}{{\arabic{Question}}{}}}\addtocounter{Question}{1}}

\newcounter{Exercice}
\setcounter{Exercice}{1}
\newcommand\Exercice{\vspace{5pt}\textbf{Exercice \arabic{Exercice}\addtocounter{Exercice}{1}.~~}}
\makeatletter
\newcommand{\Qlabele}[1]{\addtocounter{Exercice}{-1}\protected@write \@auxout {}{\string \newlabel {#1}{{\arabic{Exercice}}{}}}\addtocounter{Exercice}{1}}
\makeatother

\newif\ifCORRECTION % Test pour l'affichage des corrections
%\CORRECTIONtrue
\CORRECTIONfalse

%%%%%%%%%%%%%%%%%%%%%%%%%%%%%%%%%%%%%%%%%%%%%%%%%%%%%%%%%%%%%%%%%%%%%%%%%%%%%%%
\begin{document}\parindent0pt\parskip\smallskipamount
%%%%%%%%%%%%%%%%%%%%%%%%%%%%%%%%%%%%%%%%%%%%%%%%%%%%%%%%%%%%%%%%%%%%%%%%%%%%%%%

\begin{center}
  \textbf{\Large{Conception de sites Web}}\par\smallskip
  \textbf{\large{TP \no 2 : \og{}Un site Web personnel\fg{}}}\par\smallskip
\end{center}\bigskip

\textbf{Objectif~:~} \`{A} travers la réalisation d'un site Web personnel, 
vous affirmerez vos connaissances des langages HTML et CSS. Vous mettrez en 
\oe uvre des techniques de dynamisation à l'aide du langage Javascript côté 
navigateur et de CGI\footnote{\emph{Common Gateway Interface}} côté serveur.

%%%%%%%%%%%%%%%%%%%%%%%%%%%%%%%%%%%%%%%%%%%%%%%%%%%%%%%%%%%%%%%%%%%%%%%%%%%%%%%
\section{Introduction}\label{sec:introduction}
%%%%%%%%%%%%%%%%%%%%%%%%%%%%%%%%%%%%%%%%%%%%%%%%%%%%%%%%%%%%%%%%%%%%%%%%%%%%%%%
Le site personnel que nous vous proposons de réaliser devra respecter les 
versions HTML5 et CSS3 et sera accessible depuis l'URL :\\
\texttt{http://www.ecole.ensicaen.fr/\textasciitilde votre-identifiant/}. 

Sa réalisation s'étalera sur trois séances et demi, la moitié de la dernière 
séance étant consacrée à une présentation orale de dix minutes permettant, 
par binômes, d'en montrer le résultat en expliquant son architecture ainsi 
que les solutions que vous avez retenues pour le dynamiser.

Pourquoi par binômes alors qu'il s'agit d'un site personnel ? Et bien avant
tout pour grouper vos compétences et produire plus rapidement la structure 
générale du site ainsi que les mécanismes de dynamisation de ce dernier. Les 
ajustements personnels pourront être obtenus en modifiant la feuille de styles
(couleurs, polices utilisées, etc.) et bien entendu le contenu.  

Voyons à présent les spécifications attendues pour ce site. 

%%%%%%%%%%%%%%%%%%%%%%%%%%%%%%%%%%%%%%%%%%%%%%%%%%%%%%%%%%%%%%%%%%%%%%%%%%%%%%%
\section{Spécifications}\label{sec:specifications}
%%%%%%%%%%%%%%%%%%%%%%%%%%%%%%%%%%%%%%%%%%%%%%%%%%%%%%%%%%%%%%%%%%%%%%%%%%%%%%%
Le site Web à réaliser comportera cinq rubriques (\og{}Accueil\fg{}, 
\og{}Curriculum vit\ae\fg{}, \og{}Blog\fg{}, \og{}Portfolio\fg{},  
\og{}Contact\fg{}) qui seront accessibles depuis un menu de navigation, et
que nous décrivons plus bas.

Le site sera de type mono-page, c'est-à-dire que l'ensemble des rubriques 
accessibles sera contenu dans une page unique que vous nommerez \Code{index.html}. 
La raison de ce choix est double : d'une part vous ne disposez encore que d'un 
contenu limité qui pourra se contenter dans un premier temps d'une page unique 
et, d'autre part, nous pourrons plus facilement mettre en \oe uvre des 
techniques de dynamisation Javascript à l'intérieur de cette page.

%%%%%%%%%%%%%%%%%%%%%%%%%%%%%%%%%%%%%%%%%%%%%%%%%%%%%%%%%%%%%%%%%%%%%%%%%%%%%%%
\subsection{Structure de page}
%%%%%%%%%%%%%%%%%%%%%%%%%%%%%%%%%%%%%%%%%%%%%%%%%%%%%%%%%%%%%%%%%%%%%%%%%%%%%%%
Afin d'accéder aux différentes rubriques du site, nous imposons une structure 
en trois zones :

\begin{itemize}
  \item Un bandeau d'en-tête délimité par la balise \Code{<header>} et 
        précisant le nom du site (par exemple votre nom) et éventuellement
        un descriptif ou un slogan pour ce site, les deux associés 
        respectivement à des titres de niveaux 1 et 2. Un logo peut y être 
        ajouté. 
        Nous proposons de munir ce bandeau d'une barre de navigation délimitée
        par la balise \texttt{<nav>} afin de passer d'une rubrique à une autre. 
        Vous pouvez toutefois décider de dissocier la barre de navigation de 
        l'en-tête si vous en apportez la justification. 
        L'en-tête doit être fixe et apparaître en permanence (la barre de
        navigation aussi si vous la dissociez).  
  \item Une zone de visualisation des rubriques délimitée par la balise en
        paires \Code{<main>} et permettant d'afficher le contenu de chaque 
        rubrique. 
  \item Une barre d'état délimitée par la balise \Code{<footer>} 
        et placée préférentiellement en bas de page. Cette zone doit indiquer 
        l'information de copyright et dispose de deux liens permettant de 
        passer de la version française du site à sa version anglaise et vice 
        versa.      
\end{itemize}
\vspace{10pt}

La figure \ref{fig:mep1} propose un exemple minimal et conventionnel de ce 
qui est attendu au niveau de la mise en page de votre site. Mais rien ne 
vous empêche de sortir des sentiers battus et de proposer une mise en page
à la mesure de votre esprit créatif et qui bouscule ce conventionnel, en 
jouant, entre autres, avec les feuilles de styles. 

\begin{figure}[htbp]
  \begin{center}
    \includegraphics[width=0.7\textwidth]{http://frosch75.free.fr/snir1/pictures/site_accueil.pdf} 
    \caption{Exemple de page d'accueil pour le site Web}
    \label{fig:mep1} 
  \end{center}
\end{figure}

%%%%%%%%%%%%%%%%%%%%%%%%%%%%%%%%%%%%%%%%%%%%%%%%%%%%%%%%%%%%%%%%%%%%%%%%%%%%%%%
\subsection{Détail des rubriques}
%%%%%%%%%%%%%%%%%%%%%%%%%%%%%%%%%%%%%%%%%%%%%%%%%%%%%%%%%%%%%%%%%%%%%%%%%%%%%%%
Chaque rubrique du site mono-page correspond à une section (balise 
\Code{<section>}) munie d'un en-tête et composée ou non d'articles 
(balise \Code{<article>}). Chaque section peut être associée à une zone 
complémentaire (balise \texttt{<aside>}). 
L'en-tête de la section est composé d'un titre de niveau 1 qui est en général 
le même que le nom de la rubrique\footnote{Ceci n'est pas obligatoire si le 
visiteur peut se localiser facilement sur le site, auquel cas le titre de 
niveau 2 devient titre de niveau 1. C'est le cas sur l'exemple de la figure 
\ref{fig:mep1} où l'information de localisation est fournie par la barre 
de navigation et pourrait éviter la répétition.}, voire d'un sous-titre de 
niveau 2 et le cas échéant d'un élément de navigation. Chaque article, quant 
à lui, est composé d'au moins un titre de niveau 2, éventuellement d'un 
sous-titre de niveau 3, d'une date et de paragraphes. 
        
%%%%%%%%%%%%%%%%%%%%%%%%%%%%%%%%%%%%%%%%%%%%%%%%%%%%%%%%%%%%%%%%%%%%%%%%%%%%%%%
\subsubsection{Rubrique \Code{Accueil}}
%%%%%%%%%%%%%%%%%%%%%%%%%%%%%%%%%%%%%%%%%%%%%%%%%%%%%%%%%%%%%%%%%%%%%%%%%%%%%%%
La rubrique \Code{Accueil} est la rubrique par défaut sur laquelle se retrouve
tout nouveau visiteur (voir figure \ref{fig:mep1}). Elle contient un court 
descriptif sur vous et sur ce que les visiteurs pourront trouver sur ce site. 

%%%%%%%%%%%%%%%%%%%%%%%%%%%%%%%%%%%%%%%%%%%%%%%%%%%%%%%%%%%%%%%%%%%%%%%%%%%%%%%
\subsubsection{Rubrique \Code{Curriculum vit\ae}}
%%%%%%%%%%%%%%%%%%%%%%%%%%%%%%%%%%%%%%%%%%%%%%%%%%%%%%%%%%%%%%%%%%%%%%%%%%%%%%%
La rubrique \Code{Curriculum vit\ae} permet de visualiser votre CV. Sa 
sélection provoque l'ouverture du document que vous avez réalisé dans le TP \no 
1\footnote{Pensez à enlever du curriculum vit\ae ~vos coordonnées personnelles 
pour ne laisser éventuellement que les coordonnées professionnelles. Tout 
échange de coordonnées pourra éventuellement se faire par l'intermédiaire de 
la rubrique \Code{Contact}.}, soit dans l'onglet courant du navigateur, soit 
de manière intégrée comme l'illustre la figure \ref{fig:mep2}.

\begin{figure}[htbp]
  \begin{center}
    \includegraphics[width=0.7\textwidth]{http://frosch75.free.fr/snir1/pictures/site_cv.pdf} 
    \caption{Exemple d'intégration du curriculum vit\ae}
    \label{fig:mep2} 
  \end{center}
\end{figure}

%%%%%%%%%%%%%%%%%%%%%%%%%%%%%%%%%%%%%%%%%%%%%%%%%%%%%%%%%%%%%%%%%%%%%%%%%%%%%%%
\subsubsection{Rubrique \Code{Blog}}
%%%%%%%%%%%%%%%%%%%%%%%%%%%%%%%%%%%%%%%%%%%%%%%%%%%%%%%%%%%%%%%%%%%%%%%%%%%%%%%
La troisième rubrique (\Code{Blog}) est une rubrique permettant de vous 
exprimer sur le sujet qui vous tient le plus à c\oe ur\footnote{Un nombre 
supérieur de sujets surchargerait inutilement ce petit site.} en produisant 
des billets (articles) datés.

Cette rubrique dispose d'un fonction de mise en valeur du texte des articles 
(encadrement des articles, mise en valeur du texte à emphase, couleurs, etc.) 
par l'intermédiaire d'un bouton ou autre. 

Trois fonctions supplémentaires peuvent éventuellement être envisagées
pour ce blog : 

\begin{enumerate}
  \item Recherche et surlignage d'un mot dans l'ensemble du blog.
  \item Affichage d'un nuage des mots les plus fréquents dans le blog. 
  \item Tri des articles du blog en fonction de leurs dates (croissantes ou
        décroissantes).
\end{enumerate}
\vspace{10pt}

La figure \ref{fig:mep3} propose un exemple de mise en page de cette rubrique 
où le bloc qui précède les articles contient un champ de recherche de mots, un 
bouton de mise en valeurs des articles ainsi qu'une zone d'affichage du nuage 
des mots les plus fréquents (balise \Code{<canvas>}). 

\begin{figure}[htbp]
  \begin{center}
    \includegraphics[width=0.7\textwidth]{http://frosch75.free.fr/snir1/pictures/site_blog.pdf} 
    \caption{Exemple de rubrique \og{}blog\fg{}}
    \label{fig:mep3} 
  \end{center}
\end{figure}

Des exemples de mise en valeur des articles et de résultat d'une recherche 
sont présentés figure \ref{fig:mep4}.  

\begin{figure}[htbp]
  \begin{center}
    \includegraphics[width=0.48\textwidth]{http://frosch75.free.fr/snir1/pictures/site_blog_enjoliver.pdf}
    \includegraphics[width=0.48\textwidth]{http://frosch75.free.fr/snir1/pictures/site_blog_rechercher.pdf} 
    \caption{Exemples de mise en valeur des articles à l'aide du bouton
             \Code{Enjoliver les articles} (figure de gauche) et de résultat
             d'une recherche de mot (figure de droite).}
    \label{fig:mep4} 
  \end{center}
\end{figure}

%%%%%%%%%%%%%%%%%%%%%%%%%%%%%%%%%%%%%%%%%%%%%%%%%%%%%%%%%%%%%%%%%%%%%%%%%%%%%%%
\subsubsection{Rubrique \Code{Portfolio}}
%%%%%%%%%%%%%%%%%%%%%%%%%%%%%%%%%%%%%%%%%%%%%%%%%%%%%%%%%%%%%%%%%%%%%%%%%%%%%%%
Un portfolio est un \og{}dossier personnel dans lequel les acquis de formation
et les acquis d'expériences d'une personne sont définis et démontrés en vue 
d'une reconnaissance par un établissement d'enseignement ou un employeur.\fg{} 
(Source : http://fr.wiktionary.org/). 

Comme on peut le remarquer sur la figure \ref{fig:mep5}, les portfolios que 
l'on trouve sur le Web vont de la simple page HTML n'utilisant aucune feuille 
de styles mais permettant le téléchargement du code source des projets, à des 
pages plus complexes qui exploitent de nombreux mécanismes de dynamisation et 
proposent des démonstrations (applets Java, vidéos, etc.), mais peuvent pour
des raisons de confidentialité rester discrètes quant à leur implémentation. 

\begin{figure}[htbp]
  \begin{tabular}{cc}
    \includegraphics[width=0.49\textwidth]{http://frosch75.free.fr/snir1/pictures/exemple_portfolio_0.pdf} & \includegraphics[width=0.49\textwidth]{images/exemple_portfolio_1.pdf} \\\hline 
    \footnotesize{http://www.cs.toronto.edu/~trebla/personal/portfolio/} & \footnotesize{Projet \emph{Extended Euclidean Algorithm steps}} \\\hline 
    \includegraphics[width=0.49\textwidth]{http://frosch75.free.fr/snir1/pictures/exemple_portfolio_2.pdf} & \includegraphics[width=0.49\textwidth]{http://frosch75.free.fr/snir1/pictures/exemple_portfolio_4.pdf} \\\hline 
    \footnotesize{http://johnshipp.com/portfolio/} & \footnotesize{Projet \emph{WebPipeSSO}} \\\hline 
\end{tabular}
    \caption{Deux exemples de portfolios sur la toile. 
             \`{A} gauche la page du portfolio et à droite celle d'un des projets.}
    \label{fig:mep5} 
\end{figure}

Dans le cas présent, la rubrique \Code{Portfolio} doit démontrer aux visiteurs
vos compétences en informatique. Il sera donc nécessaire d'y mettre en avant 
les projets que vous avez menés à bien sans forcément tout y révéler. Leur mise
en avant peut nécessiter la présentation de codes sources, l'intégration de
vidéos ou des démonstrations, et donc mettre en \oe uvre des mécanismes de 
dynamisation chargées d'impacter davantage le visiteur. 

Cette rubrique est composée d'un en-tête comprenant au minimum un titre, voire
un sous-titre, et un bloc de navigation qui permet d'accéder au contenu des 
projets en les catégorisant éventuellement. 
Chaque projet est contenu dans un article daté et composé au minimum d'un titre
de niveau 2 et de paragraphes dans lesquels vous pourrez insérer du texte, des
algorithmes, des codes sources, des démonstrations, etc..

Les figures \ref{fig:mep6} et \ref{fig:mep7} illustrent un exemple de portfolio
simple où une barre de navigation permet d'accéder aux catégories de projets
ainsi qu'aux informations sur les projets eux-mêmes. 
La figure \ref{fig:mep7} propose la mise en évidence de projets avec une 
démonstration en ligne pour l'un et un affichage du code source pour 
l'autre. 

\begin{figure}[htbp]
  \begin{center}
    \includegraphics[width=0.7\textwidth]{http://frosch75.free.fr/snir1/pictures/site_portfolio1.pdf}
    \caption{Exemple de rubrique \og{}Portfolio\fg{}}
    \label{fig:mep6} 
  \end{center}
\end{figure}

\begin{figure}[htbp]
  \begin{center}
    \includegraphics[width=0.7\textwidth]{http://frosch75.free.fr/snir1/pictures/site_portfolio2.pdf} 
    \caption{Exemples de mise en évidence de projets dans la rubrique 
             \og{}Portfolio\fg{}.}
    \label{fig:mep7} 
  \end{center}
\end{figure}


%%%%%%%%%%%%%%%%%%%%%%%%%%%%%%%%%%%%%%%%%%%%%%%%%%%%%%%%%%%%%%%%%%%%%%%%%%%%%%%
\subsubsection{Rubrique \Code{Contact}}\label{sec:contact}
%%%%%%%%%%%%%%%%%%%%%%%%%%%%%%%%%%%%%%%%%%%%%%%%%%%%%%%%%%%%%%%%%%%%%%%%%%%%%%%
La rubrique \Code{Contact} fournit au visiteur un moyen de vous joindre à 
l'aide d'un formulaire. Les données envoyées par un visiteur sont reçues
par le serveur qui en retour enverra une page indiquant que la demande a bien
été prise en compte. Cette page indiquera notamment un résumé des informations  
fournies par le visiteur, le type de navigateur du visiteur ainsi que son
adresse IP puis la date. Les données transmises par le visiteur sont alors
transmise à l'administrateur du site.

Le formulaire peut éventuellement être munie d'une image 
captcha\footnote{https://fr.wikipedia.org/wiki/CAPTCHA} générée par le serveur
et qui permettra d'exclure toute demande n'ayant pu en indiquer le contenu. 

La figure \ref{fig:mep8} montre un exemple de formulaire pour la rubrique 
\Code{Contact}. En bonus, la zone de droite affiche votre localisation à l'aide 
d'une carte Google.

\begin{figure}[htbp]
  \begin{center}
    \includegraphics[width=0.7\textwidth]{http://frosch75.free.fr/snir1/pictures/site_contact.pdf} 
    \caption{Exemple de rubrique \og{}Contact\fg{}}
    \label{fig:mep8} 
  \end{center}
\end{figure}

%%%%%%%%%%%%%%%%%%%%%%%%%%%%%%%%%%%%%%%%%%%%%%%%%%%%%%%%%%%%%%%%%%%%%%%%%%%%%%%
\section{Travail à réaliser}\label{sec:travail}
%%%%%%%%%%%%%%%%%%%%%%%%%%%%%%%%%%%%%%%%%%%%%%%%%%%%%%%%%%%%%%%%%%%%%%%%%%%%%%%
%%%%%%%%%%%%%%%%%%%%%%%%%%%%%%%%%%%%%%%%%%%%%%%%%%%%%%%%%%%%%%%%%%%%%%%%%%%%%%%
La réalisation de votre site Web passe par quatre étapes : maquettage, 
structuration du contenu et mise en page, dynamisation côté client, puis
côté serveur. Les étapes sont constituées d'activités obligatoires, voire
optionnelles, classées en cinq niveaux de difficulté que nous signalons par 
des étoiles : 
\vspace{3pt}
\begin{itemize}
   \item facile \includegraphics[width=0.12\textwidth]{http://frosch75.free.fr/snir1/pictures/etoile1.pdf}
   \item plutôt facile \includegraphics[width=0.12\textwidth]{http://frosch75.free.fr/snir1/pictures/etoile2.pdf}
   \item difficulté moyenne \includegraphics[width=0.12\textwidth]{http://frosch75.free.fr/snir1/pictures/etoile3.pdf}
   \item plutôt difficile \includegraphics[width=0.12\textwidth]{http://frosch75.free.fr/snir1/pictures/etoile4.pdf}
   \item difficile \includegraphics[width=0.12\textwidth]{http://frosch75.free.fr/snir1/pictures/etoile5.pdf}
\end{itemize}
\vspace{3pt}

Vous devriez en principe franchir l'ensemble des étapes de niveaux \og{}facile\fg{}
à \og{}difficulté moyenne\fg{}.

\subsection{Maquettage}
%%%%%%%%%%%%%%%%%%%%%%%%%%%%%%%%%%%%%%%%%%%%%%%%%%%%%%%%%%%%%%%%%%%%%%%%%%%%%%%
(Niveau de difficulté \includegraphics[width=0.12\textwidth]{http://frosch75.free.fr/snir1/pictures/etoile1.pdf})

Avant de se lancer avec frénésie dans l'étape de codage du site, il est 
nécessaire de déterminer l'agencement des différents éléments dans les pages 
de manière à permettre une navigation claire et efficace.  
C'est une étape de maquettage, qui peut s'effectuer soit sur papier libre, 
soit à l'aide d'un logiciel de dessin vectoriel (outil de dessin dans 
LibreOffice, Dia, Inkscape, etc.), et qui vous permet de préciser les dimensions,
positionnements, structures supplémentaires, etc. dont vous aurez besoin pour
réaliser le site. 
Cette étape doit vous guider dans la réalisation de votre feuille de styles 
et vous amener au choix des mécanismes de dynamisation dont vous aurez besoin.

%%%%%%%%%%%%%%%%%%%%%%%%%%%%%%%%%%%%%%%%%%%%%%%%%%%%%%%%%%%%%%%%%%%%%%%%%%%%%%%
\subsection{Mise en page et contenu}
%%%%%%%%%%%%%%%%%%%%%%%%%%%%%%%%%%%%%%%%%%%%%%%%%%%%%%%%%%%%%%%%%%%%%%%%%%%%%%%
(Niveau de difficulté \includegraphics[width=0.12\textwidth]{http://frosch75.free.fr/snir1/pictures/etoile2.pdf})

Proposez les codes HTML et CSS permettant de respecter la structure attendue et
de produire la mise en page que vous avez fixée pour votre site, puis insérez-y 
votre contenu. 

Nous vous rappelons que la racine de votre site correspond au répertoire 
\Code{public\_html}. La version française de votre site Web sera contenue
dans le fichier \Code{index.html} placé à la racine et sa version anglaise, le
cas échéant, le sera dans le fichier \Code{index\_en.html}. Vos styles CSS
seront définis dans le fichier \Code{<vos\_initiales>.css} placé dans le 
sous-répertoire \Code{styles} et les images que vous utiliserez seront, 
quant à elles, stockées dans le sous-répertoire \Code{images}.

Vos codes devront être valides HTML5 et CSS3 et éviter autant que faire se peut
l'utilisation des balises universelles \texttt{<div>} et \texttt{<span>}.
Vous vérifierez le résultat sur différents navigateurs.

%%%%%%%%%%%%%%%%%%%%%%%%%%%%%%%%%%%%%%%%%%%%%%%%%%%%%%%%%%%%%%%%%%%%%%%%%%%%%%%
\subsection{Dynamisation coté client}
%%%%%%%%%%%%%%%%%%%%%%%%%%%%%%%%%%%%%%%%%%%%%%%%%%%%%%%%%%%%%%%%%%%%%%%%%%%%%%%
Vos fonctions Javascript seront définies dans le fichier \Code{<vos\_initiales>.js} 
qui sera placé dans le sous-répertoire \Code{scripts} présent à la racine du 
site. 

\subsubsection{Dynamisation du blog}
\paragraph*{Obligatoire}~\\
Réaliser la fonction Javascript
\Code{enjoliver} (\includegraphics[width=0.10\textwidth]{http://frosch75.free.fr/snir1/pictures/etoile2.pdf})
permettant de mettre en valeur les articles du blog. 

\paragraph*{Optionnel}~\\
Proposer des fonctions Javascript qui permettent de : 
\vspace{3pt}

\begin{itemize}
  \item \Code{rechercherMot} : rechercher et surligner un mot dans les articles 
        (\includegraphics[width=0.10\textwidth]{http://frosch75.free.fr/snir1/pictures/etoile3.pdf}) ;
  \item \Code{creerNuageMots} : générer un nuage des mots les plus fréquents 
        (\includegraphics[width=0.10\textwidth]{http://frosch75.free.fr/snir1/pictures/etoile4.pdf}) ;
  \item \Code{trierParDate} : trier et afficher les billets suivant leurs dates,
        croissantes
        ou décroissantes, à l'aide d'une liste déroulante des critères de tri 
        (\includegraphics[width=0.10\textwidth]{http://frosch75.free.fr/snir1/pictures/etoile3.pdf}).
\end{itemize}

\subsubsection{Dynamisation du portfolio}
\paragraph*{Optionnel}~\\
Proposer des mécanismes de dynamisation pour la rubrique \Code{Portfolio} 
en fonction du ou des projets qui y sont présentés.
\vspace{5pt}

%%%%%%%%%%%%%%%%%%%%%%%%%%%%%%%%%%%%%%%%%%%%%%%%%%%%%%%%%%%%%%%%%%%%%%%%%%%%%%%
\subsection{Dynamisation côté serveur}
%%%%%%%%%%%%%%%%%%%%%%%%%%%%%%%%%%%%%%%%%%%%%%%%%%%%%%%%%%%%%%%%%%%%%%%%%%%%%%%
Vos scripts CGI seront placés dans le sous-répertoire \Code{cgi-bin} présent 
à la racine du site, et devront être exécutables par tous. 

\subsubsection{Analyse du formulaire de contact}
\paragraph*{Obligatoire}~\\
Proposer un script CGI en bash (\Code{gerer\_form.sh}) permettant de répondre 
aux spécifications de la section \ref{sec:contact} 
(\includegraphics[width=0.12\textwidth]{http://frosch75.free.fr/snir1/pictures/etoile3.pdf}). L'envoi du mail à 
l'administrateur sera remplacé par l'écriture dans un fichier texte des 
informations fournies par le visiteur. 

\paragraph*{Optionnel}~\\
Ajouter la gestion de l'image captcha (\includegraphics[width=0.12\textwidth]{http://frosch75.free.fr/snir1/pictures/etoile3.pdf})
et insérez une carte Google pour permettre votre localisation   
(\includegraphics[width=0.12\textwidth]{http://frosch75.free.fr/snir1/pictures/etoile2.pdf}).

\subsubsection{Compteur de visites}
\paragraph*{Optionnel}~\\
Proposer un script CGI en bash (\Code{compter\_visites.sh}) permettant 
d'ajouter le nombre de visites sur le pied de page de votre site 
(\includegraphics[width=0.12\textwidth]{http://frosch75.free.fr/snir1/pictures/etoile2.pdf}). Le nombre de
visiteurs sera stocké dans le fichier \Code{visites.txt} placé à la racine.
\vspace{5pt}
\end{document}
